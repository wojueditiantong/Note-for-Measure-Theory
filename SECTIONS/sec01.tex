\section{Measure space}

%\texorpdfstring{\textbf{$\sigma$}}{} //include math expression in section
\subsection{Classes of sets}

Let \textbf{\textit{$\Omega$}} be a set.
\begin{definition}[$\pi$-system]
    Suppose \textbf{S} is a set.If $\forall A,B \in \textbf{S},A\cap B \in \textbf{S}$,we say \textbf{S} is a $\pi$-system.  
\end{definition}
   
\begin{definition}[Semi-algebra]\label{def:semi-algebra}
    A nonempty collection  \textbf{S} of subsets of $\Omega$ is called a semi-algebra on $\Omega$ if:\smallskip
    \begin{enumerate}
        \item $\Omega \in \textbf{\textit{S}}$.
        \item $A , B \in \textbf{S} \implies A \cap B \in \textbf{S} $
        \item $A \in \textbf{S} \implies A^c$ is a finite disjoint union of sets in \textbf{S}
    \end{enumerate}

    Obviously,semi-algebra  is a $\pi$-system.

\end{definition}

\begin{definition}[Algebra]\label{def:algebra}
    A nonempty collection  \textbf{S} of subsets of $\Omega$ is called a algebra on $\Omega$ if:\smallskip
    \begin{enumerate}
        \item $\Omega \in \textbf{\textit{S}}$.
        \item $A , B \in \textbf{S} \implies A \cap B \in \textbf{S} $
        \item $A \in \textbf{S} \implies A^c \in \textbf{S}$ 
    \end{enumerate}
\end{definition}

Comparing these two definitions,we can easily find that a algebra is always a semi-algebra,since $A^c = A^c \in \textbf{S}$

\begin{definition}[\texorpdfstring{\textbf{$\sigma-$}}{}algebra]\label{def:sigma-algebra}
    A nonempty collection  \textbf{S} of subsets of $\Omega$ is called a $\sigma$-algebra on $\Omega$ if:\smallskip
    \begin{enumerate}
        \item $A \in \textbf{S} \implies A^c \in \textbf{S}$
        \item $A_i \in \textbf{S}$ is a countable sequence of sets$\implies \bigcup_i A_i \in \textbf{S}$
       % \item $\Omega \in \textbf{\textit{S}}$(not necessary).
    \end{enumerate}
\end{definition}

We can easily find that a $\sigma-$algebra is always a algebra,and surely semi-algebra.
Here are some examples:\smallskip

\begin{example}[Semi-algebra]
   Let $\Omega = (0,1],\textbf{S}=\{(a,b]:0\leqslant a<b\leqslant 1\}$,then \textbf{S} is a semi-algebra on $\Omega$.
\end{example}

\begin{example}[Algebra]
    Let $\Omega = (0,1],\textbf{S}=\{\bigcup_{i=1}^n(a_i,b_i]:0\leqslant a_i<b_i\leqslant 1,n \in \N\}$,then \textbf{S} is a algebra on $\Omega$.
\end{example}

\begin{example}[\texorpdfstring{\textbf{$\sigma-$}}{}algebra]
    Let $\Omega = \{1,2\},\textbf{S}=\{\varnothing ,\{1\},\{2\},\{1,2\}\}$,then \textbf{S} is a $\sigma$-algebra on $\Omega$.
\end{example}

Since an algebra is a semi-algebra,we want to find an algebra including a specific semi-algebra.

Suppose \textbf{S} is a semi-algebra on $\Omega$,\textbf{A} is an algebra,if $\textbf{S} \subseteq \textbf{A},\textbf{A} \subseteq \textbf{F}  ,\forall \textbf{F} \in \mathcal{F} $.
$\mathcal{F} = \{\textbf{F}: \textbf{F}\ is\ an\ algebra\ on\ \Omega,\textbf{S} \subseteq \textbf{F}\}$.We call \textbf{A} is a gengerate algebra 

\begin{lemma}
    $
    \mathcal{S} \ is\ a \ semi-algebra,then \ \widetilde{\mathcal{S}}=\{finite\ disjoint\ unions\ of\ sets\ in\ \mathcal{S} \}
    \\
     is \ an \ algebra,\ called\ the\ \textbf{algebra}\ \textbf{generated}\ by\ \mathcal{S}.$
\end{lemma}
\begin{proof}[Proof]
    Suppose $A=\sum_{i} S_i,B=\sum_{j} T_j \in \widetilde{\mathcal{S}} $, we assume the index sets are finite.then $A \cap B \in \widetilde{\mathcal{S}},$ The definition of $\mathcal{S}$ implies $S_i^c \in \widetilde{\mathcal{S}}$ since $S_i^c=\sum_{k}C_k$ where $C_k \in \mathcal{S}$
\end{proof}



\subsection{Measure}
A measure $\mu$ on a set $(X,\mathcal{B})$ is a function $\mu: \mathcal{B} \rightarrow [0,\infty]$ that satisfies the following conditions:

$\mu(\phi) = 0$
If $(E_n)$ is a sequence of disjoint sets in $\mathcal{A}$, then $\mu\left(\bigcup_{n=1}^\infty E_n\right) = \sum_{n=1}^\infty \mu(E_n)$
\subsection{Some related discussion}

Someone may have taken the class of toplogy,and he may find a toplogy "looks like" a $\sigma$-algebra.Now let's recall its definition:\\ [10pt]
\begin{definition}[topology]
    A topology on a set X is a collection $\O$ of subsets of X satisfying
the following properties:\smallbreak
\begin{enumerate}
    \item $\varnothing \in \O $ and $ X \in \O$
    \item $A_i \in \O$ is a \textbf{arbitrary} sequence of sets$\implies \bigcup_i A_i \in \O$
    \item $A_i \in \O$ is a finite sequence of sets$\implies \bigcap_i A_i \in \O$
\end{enumerate}
\end{definition}
    The elements $O \in \O$ are called open sets and the pair (X, $\O$) is called a topological
space.

    Although a $\sigma$-algebra looks like a toplogy with defining the sets in it as open sets,but 
we must point that $\sigma$-algebra is only close under countable union while a topology is close under
arbitrary union,that's why a $\sigma$-algebra may not be a toplogy,

\begin{example}
    The Borel sets in [0,1], obviously. It contains all points but not all subsets.
\end{example}
    See more on \href{https://math.stackexchange.com/questions/2887034/sigma-algebra-v-s-topology}{sigma-algebra-v-s-topology}

\newpage
\subsection*{Problems}
\addcontentsline{toc}{subsection}{Problems}

\begin{enumerate}
    \item Show that the interaction of two $\sigma$-algebra is a $\sigma$-algebra.\smallskip
    \item Show that the union of two $\sigma$-algebra is never a $\sigma$-algebra unless one is included in another.\smallskip
    \item 
\end{enumerate}

