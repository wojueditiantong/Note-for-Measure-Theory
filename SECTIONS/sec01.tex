\section{Measure space}

%\texorpdfstring{\textbf{$\sigma$}}{} //include math expression in section
\subsection{Classes of sets}

Let \textbf{\textit{$\Omega$}} be a set.

\begin{definition}[Semi-algebra]\label{def:semi-algebra}
    A nonempty collection  \textbf{S} of subsets of $\Omega$ is called a semi-algebra on $\Omega$ if:\smallskip
    \begin{enumerate}
        \item $\Omega \in \textbf{\textit{S}}$.
        \item $A , B \in \textbf{S} \implies A \cap B \in \textbf{S} $
        \item $A \in \textbf{S} \implies A^c$ is a finite disjoint union of sets in \textbf{S}
    \end{enumerate}
\end{definition}

\begin{definition}[Algebra]\label{def:algebra}
    A nonempty collection  \textbf{S} of subsets of $\Omega$ is called a algebra on $\Omega$ if:\smallskip
    \begin{enumerate}
        \item $\Omega \in \textbf{\textit{S}}$.
        \item $A , B \in \textbf{S} \implies A \cap B \in \textbf{S} $
        \item $A \in \textbf{S} \implies A^c \in \textbf{S}$ 
    \end{enumerate}
\end{definition}

Comparing these two definitions,we can easily find that a algebra is always a semi-algebra,since $A^c = A^c \in \textbf{S}$

\begin{definition}[\texorpdfstring{\textbf{$\sigma-$}}{}algebra]\label{def:sigma-algebra}
    A nonempty collection  \textbf{S} of subsets of $\Omega$ is called a $\sigma$-algebra on $\Omega$ if:\smallskip
    \begin{enumerate}
        \item $\Omega \in \textbf{\textit{S}}$.
        \item $A \in \textbf{S} \implies A^c \in \textbf{S}$
        \item $A_i \in \textbf{S}$ is a countable sequence of sets,$\implies \cup_i A_i \in \textbf{S}$
    \end{enumerate}
\end{definition}

We can easily find that a $\sigma-$algebra is always a algebra,and surely semi-algebra,namely:\smallskip



